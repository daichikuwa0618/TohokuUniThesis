%%%%%%%%%%%%%%%%%%%%%%%%%%%%%%%%%%%%%%%%%%%%%%%%%%%%%%%%%%%%%%%%%%%%%%
%%% conlusion.tex
%%% 結論
%%%%%%%%%%%%%%%%%%%%%%%%%%%%%%%%%%%%%%%%%%%%%%%%%%%%%%%%%%%%%%%%%%%%%%
%%%    「学位論文用TeXフォーマット」
%%%
%%%%%%%%%%%%%%%%%%%%%%%%%%%%%%%%%%%%%%%%%%%%%%%%%%%%%%%%%%%%%%%%%%%%%%
\chapter{結論}
\label{chap:conclusion}
%%%%%%%%%%%%%%%%%%%%%%%%%%%%%%%%%%%%%%%%%%%%%%%%%%%%%%%%%%%%%%%%%%%%%%
%%%  main.texから\include : YaTeX用コマンド
%#!platex main.tex
%%%%%%%%%%%%%%%%%%%%%%%%%%%%%%%%%%%%%%%%%%%%%%%%%%%%%%%%%%%%%%%%%%%%%%

本研究ではALE人工流れ星を対象として,軌道計算及び流体場解析を行った.
軌道計算は先行研究と同様の手法を用いて行い,
特に流星源速度変化及び流星源質量変化の分布において良好な一致を示した.
さらに粘性係数算出法を改善することでより正確な軌道計算を行うことが可能になった.

次に軌道計算で得られた軌道上において,
最小Knudsen数を取る高度73~km地点での流体場解析を行った.
解析手法としてはNavier–Stokes方程式を解くNavier–Stokes解析と,
DSMC法を用いた.
Navier–Stokes解析により得られた局所Knudsen数を見ると,
ほとんどの領域でKnudsen数は1を超え,
連続体近似が破綻していることが示唆された.
また,DSMC法との比較では$x$方向および並進温度において両者は大きく異る分布を示した.
さらに,淀み点熱流束により流体場解析結果を比較した.
比較対象として軌道計算で用いたDKRモデルを使用した.
Navier–Stokes解析での淀み点熱流束はDKRモデルの87\%だったのに対して,
DSMC法での淀み点熱流束はDKRモデルの49\%に過小評価された.
DKRモデルは連続体を仮定したモデルであり,
人工流れ星の軌道のように高高度に適用することは困難であると考えられる.
また,DSMC法とDKRモデルでの淀み点熱流束が大きく異なったことから,
実際に流星源が受ける加熱率は本研究で示したものより小さくなっており,
これによって軌道が変わってしまう可能性が考えられる.

最後に,軌道計算とDSMC法の連成解析を行った.
本研究ではKnudsen数と抗力係数・淀み点熱流束にそれぞれ相関があると仮定した.
まず,Knudsen数のオーダーが切り替わる地点でDSMC法による流体場解析を行い,
DSMC法で得られた抗力係数とHendersonのモデル式~\cite{henderson1976drag}を比較することで,
人工流れ星が移動する軌道に適用する補正テーブルを作成した.
抗力係数の連成解析では,淀み点熱流束と異なり,どのKnudsen数においても約80\%以上一致した.
この結果を用いて抗力係数を補正し,軌道計算を行ったところ,
得られた軌道は補正を行わない場合とほぼ一致し,
抗力係数はHendersonのモデル式~\cite{henderson1976drag}に従うことが分かった.

%
抗力係数と同様にしてDSMC法との連成解析により淀み点熱流束の補正を行った.
DSMC法により得られる淀み点熱流束とDKRモデルでの淀み点熱流束を比較することで補正テーブルを作成した.
作成した補正テーブルを見ると,淀み点熱流束の補正係数はKnudsen数と相関係数$-0.9947$の負の相関があることがわかった.
さらに,線形近似を行うことで淀み点熱流束を補正する関数を定義し,
補正を加えた軌道計算を行ったところ,軌道は大きく変化し,
特に高度100~km以上の高高度における質量減少の様子が異なることが分かった.
また,最大発光地点を補正のあり・なしで比較を行ったところ,
51~kmの差異が生じた.
発光地点から半径200~km以内での観測を想定したALE人工流れ星にとって,
約50~kmの差異は無視できるものではなく,
観測地点のより厳格な検証を行う必要性が示唆された.
%
%

この連成解析によって,遷移領域においても,
DSMC法による正確な淀み点熱流束を見積りを反映した軌道計算を行うことが可能になり,
以前のものよりも高精度な軌道計算を達成した.

今後は,アブレーションを考慮したDSMC法による解析~\cite{bariselli2020aerothermodynamic}を行い,
発光効率の見積もりを行うことで,
流星の発光メカニズムを調査する.